\documentclass[11pt]{article}
\usepackage{graphicx} 
\usepackage{algorithm2e}
\usepackage{mathtools}
%Gummi|065|=)
\title{\textbf{ASim: Algorand Simulator}\\ {\normalsize CS620 Project}}
\author{Aman Jindal, 173050025, jindalaman@cse.iitb.ac.in\\
		Harikrishnan R, 183050011, harikrishnan@cse.iitb.ac.in}

\date{}
\usepackage{graphicx}
\begin{document}

\maketitle
\graphicspath{{images/}}
\section{Introduction}
Algorand is a blockchain/cryptocurrenncy protocol that makes use of proof of stake. It is a permissionless system that support a huge scale, good throughput, and transaction finality. As in the case of other proof of stake protocols, Algorand uses a certain amount of randomization and the stake to decide the next leader. This project is an implementation of the Algorand protocol.\\ 

The project was implemented in Python 3 and makes use of the following libraries:
\begin{itemize}
	\item simpy: a framework for implementing a discrete event simulator. This was used to set up the discrete event simulator by using an 'environment' that is passed around in all the functions to simulate passing of time.
	\item numpy: to obtain the required random distributions (uniform, gaussian).
	\item fastecdsa: a faster implementation of Elliptic Curve DSA used for messsage signatures.
	\item scipy.stats: To easily compute the binomial sum in sortition.
	\item random: For different (pseudo-) random numbers involved.
	\item hashlib: To generate SHA256 hash of the blocks.
	\item time: To measure the execution time for debugging etc.
\end {itemize}


\section{The Algorand Algorithm}
The algorithm that we implemented was as faithful to the Algorand algorithm as possible. The heart of the algorithm is a sortition function that makes use of the 
\section{The Attack}
A side channel attack is used to obtain the SM sequence corresponding to a known message, signature pair. If the SM sequence corresponding to an ephemeral key is known, then we can deduce part of the ephemeral key.\\

The attack collects a sufficient number of message, signature pairs (m$_i$, (r$_i$, s$_i$))
The paper focuses on the case where two substrings of bits in each ephemeral key are known. 

The paper goes into several substitutions to get the following basis matrix, M:


and v = ( 0, 0, ..., 0, v$_1$, v$_2$, ..., v$_{n-1}$ ) is the target (non-lattice) vector.
Let b = (b$_0$, b$_1$, ..., b$_n$, d$_0$, d$_1$, ..., d$_n$) be the vector to be computed where the pair $<$ di, bi $>$ constitute the unknown blocks of the ith  ephemeral key. 

Thus we have w - v = b. The Closest Vector Problem is to find the lattice vector w. 

Since, the positions and the lengths of the known blocks would be different across keys, therefore, B is multiplied by a weight vector, D.

\pagebreak
The CVP can be reduced to an SVP instance. The new basis matrix B’ would be of dimension 2n + 1. \\


where D is a 2n x 2n diagonal matrix 

This is finally fed into an SVP solver.
Result of the SVP instance will help us compute the DSA secret key.

\section{Our Implementation}

Instead of actually performing a side channel attack, we simulate the DSA parameters. Using Python’s Crypto library, we generate DSA parameters
Then we generate a number of ephemeral keys, hash and the DSA signature corresponding to some message. Each ephemeral key is converted into SM sequence and then from the SM sequence we obtain a bit of string in which we can guess some of the bits\\



\pagebreak

We then used the following algorithm which helped with key retrieval in a \textbf{noise free} channel.\\


\begin{algorithm}[H]
\SetAlgoLined
1. Sort S on the basis of the sum of the lengths of the interior block and the rightmost block.

2. Let $\gamma_min$ be the minimum integer such that the total number of leaked bgits in the top $\gamma_min$ EKOs in S is $>=$ 160

3. i = $\gamma_min$

4. maxIter = $|$S$|$

 \While{i $<=$ maxIter AND keyNotFound = true}{
  Create an SVP instance using the top i EKO's from S
  
  Solve the SVP instance, compute the DSA private key and verify it's correctness
  
  \eIf{computed key is correct}{
 keyNotFound = false
   }{
   i = i + 1
  }
 }
 \caption{Noise-free key retrieval}
\end{algorithm}

\medskip


\medskip


\medskip


\medskip

To obtain the DSA signing key, we draw the top n EKOs. We build an SVP instance
from the input provided by these EKOs and solve the SVP problem. Then we compute the
DSA key and verify its correctness. If it's not correct, we continue this process but we now
include the next EKO from S while retaining the existing EKOs. This is repeated
until we obtain the correct DSA key.

\pagebreak

Then, we used the following algorithm to implement the same for a noisy channel. \\

\begin{algorithm}[H]
\SetAlgoLined
1. Populate A with the fewest number of randomly selected EKOs from T such that the total number of bits leaked from the EKOs is A $>$160.

2. T$`$ = T - A

 \While{keyNotFound = true AND $|$A$|$ $<= \gamma_{max}$}{
  Create an SVP instance using the EKO's from A
  
  Solve the SVP instance, compute the DSA private key and verify it's correctness
  
  \eIf{computed key is correct}{
 keyNotFound = false
 
 \textbf{return} computed DSA key and A
   }{
   Select a ranom EKO e from T$`$
   
   A = A $\cup$ \{e\}
   
   T$`$ = T$`$ - \{e\}
  }
 }
3. Repeat till key is not found and iter $<$ iterMax\\


\medskip


 \caption{Noisy channel key retrieval}
\end{algorithm}

\end{document}. 